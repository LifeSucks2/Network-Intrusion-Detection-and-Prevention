Die Informationstechnologien (IT) beeinflussen quasi jede fassette des modernen Lebens. Die Anforderungen werden von Tag zu Tag mehr und diese werden mithilfe von modernen IT-Systemen unterstützt. Heute ist die Technologie bzgl. IT fortgeschritten. Schnelle Prozessoren, schnelles Netzwerk sowie Corona bedingt arbeiten aus dem eigenen Zuhause. Smart Devices die verschiedene Protokolle verwenden um mit dem internen und externen Netzwerk zu kommunizieren bringen sowhol Nutzen als auch kritische Sicherheitslücken mit sich. Kurz gesagt, mit dem Fortschritt entstehen auch immer mehr Schwachstellen. Diese müssen beachtet und geschützt werden. In dieser wissenschaftlichen Arbeit werden Sicherheitsrelevante Probleme erkannt und mögliche Schutzmechanismen vorgestellt. Angriffe auf die Vertraulichkeit, Integrität sowie Verfügbarkeit der Daten sind ein Problem und dürfen nicht auf die leichte Schulter genommen werden. Der Man-in-the-Middle Angriff verletzt alle drei Schutzziele und ist ein kritisch anzusehendes Problem dessen Lösung gesucht, weitererforscht sowie umgesetzt werden muss. Unternehmen die Ihr Geld mit Webppplikationen verdienen sind mit einem Angriff auf die Verfügbarkeit Ihres Dienstes gezwungen das Lösegeld zu zahlen. Das muss so gut es geht unterbunden werden. Deshalb ist ein DoS/DDoS ein wichtiger Aspekt, welches hier auch unter die Lupe genommen wird.  

\clearpage