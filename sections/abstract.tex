\begin{abstract}
Seit den frühen Anfängen des Internets, sogenannte "Internet Epidemien" haben weltweit enormen Schaden verursacht und gefährden die Sicherheit der Systeme bis heute noch.
Als Internet Epidemien bezeichnet man bösartige Software die sich selbständig über das Internet verteilen kann. Als Beispiel gab es 1988 den sogenannten "Morris worm". Dieser schaffte es, 10\% aller Internet Hosts zu infizieren. Im Jahre 2001 schaffte es der "Code Red Worm" mehr als 350.000 Internet Hosts innerhalb eines Tages zu kompromittieren.
Network Intrusion Detection ist ein neuartiges Vorgehen, um in bestehenden Computern oder Netzwerken eine Art Sicherheit darzustellen, während diese weiterhin in ihrem "offenen" Modus operieren können.
Das Ziel der Intrusion Detection ist es, wenn möglich in Echtzeit, unautorisierte Zugriffe oder ein Misshandeln des Systems zu identifizieren. 

Ein Grund, warum Intrusion Detection heutzutage eine schwierige Aufgabe darstellt, ist die Vermehrung heterogener Netwerke durch die vermehrte Konnektivität von Computersystemen. Dadruch wird es Angreifern einfacher gemacht, auf diese Systeme zuzugreifen.

Intrusion Detection Systems (IDS) basieren auf der Vermutung, dass unautorisierte Zugriffe erkennbar sind, als auch das Verhalten eines Angreifers stark unterscheidbar von dem eines tatsächlichen Nutzers ist.


\keywords{Network Intrusion  \and Network Intrusion Detection \and Hacking.}
\end{abstract}

\clearpage